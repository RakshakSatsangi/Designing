\documentclass[10pt]{article}


\begin{document}

\title{README}
\author{Aviral Jain : 2013CS10215\\Rakshak Satsangi : 2013CS10250\\
}
\date{\today}
\maketitle

	\section{Compile and run}
		Use the command make execute NUMTHREADS=n (n is the number of balls) in the submission directory to compile and run the source codes.
		
		Use make or make all to compile the source codes and generate the executable file makea.
		
		Use make maindebug NUMTHREADS=n to run the gdb debugged version.
		
		Use make debugg NUMTHREADS=n to run the debugged version made by us.
		
		Use make 3dim to run the 3D model made by us.
		
		Use make doc to generate all PDF Files.



	\section{Use the application}
		Press SPACEBAR to freeze the position of the balls.
		
Press ESC to get menu screen.

On the menu screen, click on SHOW PAUSED SCREEN to get the screen in
frozen position. (same as using SPACEBAR)

To increase speed of any ball, left click on the ball on either the frozen or running screen.

To decrease speed of any ball, right click on the ball.

To stop a ball, click with the middle mouse button or wheel on the ball.

Click on INSTRUCTIONS in the menu to get the instructions screen.

Click anywhere or press ESC on the instructions screen to get the menu screen.

Click RESUME on the menu screen to get back to the main screen with running
balls.

Press 'f' to enter/exit FULL screen Mode.

Press 'q' to Exit the Window.

Keep 'w'/'a'/ 'd'/ 'x' pressed to enforce gravity in top/left/right/bottom walls
	


\end{document}